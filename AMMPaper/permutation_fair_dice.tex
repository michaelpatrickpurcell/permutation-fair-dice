\RequirePackage{fix-cm}
\documentclass{article}
\usepackage{maa-monthly}

%% IF YOU HAVE FONTS INSTALLED
%\usepackage{mtpro2}
%\usepackage{mathtime}
\usepackage{mathptmx}

\usepackage{booktabs}
\usepackage{caption}
\usepackage{multirow}
\usepackage{lipsum}

\theoremstyle{plain}
\newtheorem{theorem}{Theorem}
\newtheorem{proposition}[theorem]{Proposition}
\newtheorem{lemma}[theorem]{Lemma}
\newtheorem{corollary}[theorem]{Corollary}

\theoremstyle{definition}
\newtheorem*{definition}{Definition}
\newtheorem*{remark}{Remark}
\newtheorem*{example}{Example}

\title{Permutation Fair Dice}
\author{Michael Purcell }
\date{January 2023}

\begin{document}

\maketitle

\begin{abstract}
\lipsum[3]   
\end{abstract}

\section{Introduction}
Many tabletop games are designed so that players take \emph{turns} interacting with the game's mechanisms. As such, the players need some way to determine the order in which they will take their turns, i.e. the \emph{turn order}. For some games, the turn order can significantly affect how the players will play the game and the players will need a way to determine the turn order that is \emph{fair}. We will say a protocol is fair if every possible turn order is produced with equal probability.

% For two-player games, this is a straightforward task. The players can simply agree to identify each player with one side of a coin and use the result of a toss of that coin to determine which player will go first. Thereafter, the players can simply alternate turns.

% For games with more than two players, the problem is more complicated. To understand why, consider a natural generalization of the strategy described above for two-player games. For a game with $n$ players, the players can agree to identify each player with one face of an $n$-sided die and use the result of a roll of that die to determine which player will go first. 
% Unlike in the two-player game, this is not sufficient to completely determine the turn order.  In particular, the remaining players do not know the order in which they should take their turns. In principle, these players could again use a die to determine which of them should go next. This could continue, specifying the placement for one player at a time, until the turn order is completely specified. 

One option is to have each player roll a standard die. If no two players roll the same number then the results of their rolls can be used to determine a turn order, perhaps by specifying that players should take turns in descending order according to the values of their rolls. If two or more players do roll the same number, then the players will have to do more to determine a complete turn order. A natural choice is to have have the tied players roll their dice again and use the new results to break the tie. While effective, this kind of multi-round protocol can be a bit cumbersome.

Suppose that, instead of using standard dice, each player uses a die from a set of nonstandard dice with the following properties:
\begin{itemize}
    \item No number appears on more than one face of any die in the set, so rolling this set of dice always yields a result with $n$ distinct numbers. 
    \item The face values are arranged such that if all $n$ dice are rolled and then sorted according to the resulting values, then all $n!$ permutations of the dice are equally likely.
\end{itemize}
%First, no number appears on more than one face of any die in the set, so rolling this set of dice always yields a result with $n$ distinct numbers. Second, the numbers are arranged on the faces in such a way that if all $n$ dice are rolled and then sorted according to the resulting values, then all $n!$ possible permutations of the dice are equally likely to occur.
We will say that a set of dice with these properties is \emph{permutation fair}. By using a set of dice that is permutation fair, the players can determine the turn order for their game in a single round. Table \ref{table:4of4_perm_fair_dice} gives one example of such a set that was discovered by Eric Harshbarger and Robert Ford in 2010 \cite{harshbarger2010}.

% Unfortunately, these kinds of multi-round protocols are cumbersome.
% Many groups of players instead simply roll a die to determine which of them will go first, and then take their turns in the arbitrary  order determined by how they are seated around the table. This protocol has a number of flaws. Most obviously, if Alice and Bob are friends and therefore usually sit next to one another, then some turn orders will be more likely than others. For some games, such a turn order might be said to be ``unfair''.

% \begin{figure}[h]
%     \captionof{table}{A set of three 6-sided dice that are permutation fair.} \label{table:3of3_perm_fair_dice}
%     \medskip
%     \centering
%     \begin{tabular}{lrrrrrr} \toprule
%         \multirow{2}[1]{*}{Die}  & \multicolumn{6}{c}{Faces} \\ \cmidrule(l){2-7}
%           & \romannumeral 1 & \romannumeral 2 & \romannumeral 3 & \romannumeral 4 & \romannumeral 5 & \romannumeral 6 \\ \midrule
%         $A$ & 1 & 6 & 9 & 10 & 14 & 17 \\
%         $B$ & 2 & 5 & 7 & 12 & 15 & 16 \\
%         $C$ & 3 & 4 & 8 & 11 & 13 & 18 \\ \bottomrule   
%     \end{tabular}
% \end{figure}

\begin{figure}[h]
    \captionof{table}{A set of four 12-sided dice that are permutation fair. }\label{table:4of4_perm_fair_dice}
    \medskip
    \centering
\begin{tabular}{c rrrrrrrrrrrr} \toprule
\multirow{2}[2]{*}{Die} &  \multicolumn{12}{c}{Faces} \\ \cmidrule(lr){2-13}     
   & \romannumeral 1 & \romannumeral 2 & \romannumeral 3 & \romannumeral 4 & \romannumeral 5 & \romannumeral 6 & \romannumeral 7 & \romannumeral 8 & \romannumeral 9 & \romannumeral 10 & \romannumeral 11 & \romannumeral 12 \\ \midrule
A & 1 & 8 & 11 & 14 & 19 & 22 & 27 & 30 & 35 & 38 & 41 & 48 \\
B & 2 & 7 & 10 & 15 & 18 & 23 & 26 & 31 & 34 & 39 & 42 & 47 \\
C & 3 & 6 & 12 & 13 & 17 & 24 & 25 & 32 & 36 & 37 & 43 & 46 \\
D & 4 & 5 & 9 & 16 & 20 & 21 & 28 & 29 & 33 & 40 & 44 & 45 \\ \bottomrule
\end{tabular}
\end{figure}


% This set of dice has two useful properties.  First, no number appears more than once, so rolling this set of dice always yields a result with three distinct numbers. Second, the numbers are arranged on the faces in such a way that if all three dice are rolled and then sorted according to the resulting values, then all six possible permutations of the dice are equally likely to occur. As such, we say that both this set of dice and the turn order protocol based on this set of dice are \emph{permutation fair}.




% Consider instead the following protocol for a three-player game that uses the set of nonstandard dice described in Table \ref{table:3of3_perm_fair_dice}.
% To use this set of dice to determine the turn order, each player takes a single die and rolls it. The players will take their turns in descending order according to the values of their rolls.

% \begin{example}
% Alice ($A$), Bob ($B$), Charlie ($C$), and David ($D$) want to use the protocol based on the set of dice described in Table \ref{table:4of4_perm_fair_dice} to generate a turn order. If their rolls are $A: 22$, $B: 7$, $C: 36$, and $D: 16$ then the turn order would be $C > A > D > B$.
% \end{example}




\end{document}