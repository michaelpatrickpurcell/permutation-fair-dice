\documentclass[aspectratio=169]{beamer}

\usetheme{sii}
%\usepackage{cmbright}
\usepackage{arev}
%\usepackage{kerkis}
%

\setsansfont{Bitstream Vera Sans}
\newfontfamily\vera{Bitstream Vera Sans}
\setbeamerfont{title}{size=\Huge, family=\vera}
\setbeamerfont{frametitle}{size=\huge, family=\vera}

\setmonofont{Bitstream Vera Sans Mono}

\usepackage[thicklines]{cancel}
\renewcommand{\CancelColor}{\color{siiblue}}
\usepackage{stmaryrd}
\usepackage[theorems]{tcolorbox}
\newtcbtheorem[use counter=theorem]{siitheorem}%
  {Theorem}{fonttitle=\upshape\large, fontupper=\slshape,
    colback=white, colframe=siibrown, colbacktitle=white, coltitle=siiorange}{theorem}

\newtcbtheorem[use counter=theorem]{siiproposition}%
  {Proposition}{fonttitle=\upshape\large, fontupper=\slshape,
    colback=white, colframe=siibrown, colbacktitle=white, coltitle=siiorange}{definition}

\newtcbtheorem[use counter=theorem]{siilemma}%
  {Lemma}{fonttitle=\upshape\large, fontupper=\slshape,
    colback=white, colframe=siibrown, colbacktitle=white, coltitle=siiorange}{definition}

\newtcbtheorem[use counter=theorem]{siicorollary}%
  {Corollary}{fonttitle=\upshape\large, fontupper=\slshape,
    colback=white, colframe=siibrown, colbacktitle=white, coltitle=siiorange}{definition}

\newtcbtheorem[use counter=theorem]{siiconjecture}%
  {Conjecture}{fonttitle=\upshape\large, fontupper=\slshape,
    colback=white, colframe=siibrown, colbacktitle=white, coltitle=siiorange}{definition}

\newtcbtheorem[use counter=theorem]{siidefinition}%
  {Definition}{fonttitle=\upshape\large, fontupper=\slshape,
    colback=white, colframe=siibrown, colbacktitle=white, coltitle=siiorange}{definition}

\newtcbtheorem{siiproof}%
  {Proof}{fonttitle=\upshape\large, fontupper=\slshape,
    colback=white, colframe=white, colbacktitle=white, coltitle=siiorange}{definition}


\usepackage{booktabs}
\usepackage{multirow}

\title{Permutation Fair Dice}
\author{Michael Purcell}
\date{December 2022}

\begin{document}
\small
\begin{frame}[Triangle=siiorange]
\titlepage
\end{frame}

\section{Prior Art}
\begin{frame}[Triangle=siiorange]
	\tocpage
\end{frame}

\begin{frame}[triangle=siiblue]
\frametitle{Go First Dice}
In 2010 Robert Ford and Eric Harshberger discovered a set of four 12-sided dice that they called "Go First Dice".

\vfill

These dice are non-standard dice and are numbered such that when rolled together:
\begin{itemize}
	\item No two dice will ever show the same value.
	\item When sorted according to the values shown on the faces every permutation of the dice is equally likely.
\end{itemize}
\end{frame}

\begin{frame}[triangle=siiblue]
\frametitle{Go First Dice Details}
The numbers on the faces of the Go First Dice are:
{
\footnotesize
\begin{table}
\begin{tabular}{c rrrrrrrrrrrr} \toprule
\multirow{2}[2]{*}{Die} &  \multicolumn{12}{c}{Faces} \\ \cmidrule(lr){2-13}     
   & i & ii & iii & iv & v & vi & vii & viii & ix & x & xi & xii \\ \midrule
A & 1 & 8 & 11 & 14 & 19 & 22 & 27 & 30 & 35 & 38 & 41 & 48 \\
B & 2 & 7 & 10 & 15 & 18 & 23 & 26 & 31 & 34 & 39 & 42 & 47 \\
C & 3 & 6 & 12 & 13 & 17 & 24 & 25 & 32 & 36 & 37 & 43 & 46 \\
D & 4 & 5 & 9 & 16 & 20 & 21 & 28 & 29 & 33 & 40 & 44 & 45 \\ \bottomrule
\end{tabular}
\end{table}
}
\end{frame}

\begin{frame}[triangle=siiblue]
\frametitle{Three-Player Go First Dice}
It is also possible to construct a set of three 6-sided dice that are permutation fair. 

\vfill

The numbers on the faces of one such set are:
{
\footnotesize
\begin{table}
\begin{tabular}{c rrrrrr} \toprule
\multirow{2}[2]{*}{Die} &  \multicolumn{6}{c}{Faces} \\ \cmidrule(lr){2-7}     
   & i & ii & iii & iv & v & vi \\ \midrule
A & 1 & 5 & 10 & 11 & 13 & 17 \\
B & 3 & 4 & 7 & 12 & 15 & 16 \\
C & 2 & 6 & 8 & 9 & 14 & 18 \\ \bottomrule
\end{tabular}
\end{table}
}
\end{frame}

\begin{frame}[triangle=siiblue]
\frametitle{Five Player Go First Dice?}
Notice that for a set $S$ of $n$ $m_s$-sided dice to be permutation fair,  the number of possible outcomes must be divisible by the number of permutations on $n$ elements.

\vfill

That is, we must have $n! \ \vert \ m_s^n$.

\vfill

In particular, this means that for any set of five $m_s$-sided permutation fair dice we must have $ 30 \ \vert \ m_s$.

\vfill

The smallest value of $m_S$ such that a set of five $m_s$-sided permutation fair dice exists is unknown.

\end{frame}

\begin{frame}[triangle=siiblue]
\frametitle{Larger Face Counts}
Eric Harshberger has discovered several sets of five-player permutation fair dice with more than thirty faces per die.

\vfill

Most of those sets are comprised of dice with different numbers of faces.  The non-homogeneous dice can be homogenized by duplicating the faces on each die to create a set of $m$-sided dice where $m$ is the least common multiple of the number of faces on each of the original dice.

\vfill

The best such set of dice that he has found can be realized as a set of five 180-sided dice.
\end{frame}

\section{My Contributions}
\begin{frame}[Triangle=siiorange]
	\tocpage
\end{frame}


\subsection{Terminology and Notation}
\begin{frame}[triangle=siiblue]
\frametitle{Dice as Strings}
A common convention in other works on non-standard dice is to represent each set of dice as a string.

\vfill

A set of $n$ dice is represented by a string comprised of $n$ distinct characters.  

\vfill

The values on the faces of the dice are assigned in according to the indices of the corresponding symbols in the string.
\end{frame}

\begin{frame}[triangle=siiblue]
\frametitle{String Representation Example}
For example, consider the three-player permutation fair dice with face values given by:
{
\footnotesize
\begin{table}
\begin{tabular}{c rrrrrr} \toprule
\multirow{2}[2]{*}{Die} &  \multicolumn{6}{c}{Faces} \\ \cmidrule(lr){2-7}     
   & i & ii & iii & iv & v & vi \\ \midrule
A & 1 & 6 & 8 & 11 & 15 & 16 \\
B & 2 & 5 &  9 & 10 & 13 & 18 \\
C & 3 & 4 & 7 & 12 & 14 & 17 \\ \bottomrule
\end{tabular}
\end{table}
}
\vfill

This set of dice can also be represented by the string:
\begin{equation*}
		\texttt{abccba} \ \texttt{cabbac} \ \texttt{bcaacb}
\end{equation*}
\end{frame}


\begin{frame}[triangle=siiblue]
\frametitle{Go First String}
The string representation of Go First Dice is:
\begin{equation*}
	\texttt{abcddcba} \ \texttt{dbaccabd} \ \texttt{cbaddabc} \  \texttt{cbaddabc} \ \texttt{dbaccabd} \ \texttt{abcddcba}
\end{equation*}
We'll call this string the \emph{Go First String}.

\vfill

Notice that:
\begin{itemize}
	\item The Go First String is a palindrome.  
	\item Both halves of the Go First String can be decomposed into three blocks, each of which is itself a palindrome.
\end{itemize}
\end{frame}

\begin{frame}[triangle=siiblue]
\frametitle{Strings as Dice}
We can use the string representation of a set of dice to compute the probability of any outcome when the corresponding dice are rolled.

\vfill

In particular, we can compute the number of ways that each pattern can occur.

\vfill

We will write $(\mathbf{x})_s$ to denote the number of times that that the pattern $\mathbf{x}$ appears in the string $s$.

\end{frame}

\begin{frame}[triangle=siiblue]
\frametitle{$\ell/n$ Permutation Fairness}
\begin{siidefinition}{$\ell/n$ Permutation Fairness}{}
	A set of dice $S$ is $\ell/n$ permutation fair if $|S| = n$ and every subset $T \subset S$ with $|T| \leq \ell$ is permutation fair.
\end{siidefinition}

\vfill

\begin{siilemma}[label=perm_fair_counts]{}{}
Let $S$ be a set of $n$ $m_s$-sided dice with string representation $s$.  $S$ is $\ell/n$ permutation fair if and only if
\begin{equation*}
(\mathbf{x})_s = \frac{m_s^\ell}{\ell!}.
\end{equation*}
for all $\mathbf{x} \subset s$ with $|\mathbf{x}| = \ell$.
\end{siilemma}
\end{frame}

\begin{frame}[triangle=siiblue]
\frametitle{Concatenating Dice}
\begin{siidefinition}{}{}
Let $S$ and $T$ be sets of dice with string representations $s$ and $t$ respectively.   We define $S \Vert T$ to be the set of dice with string representation $s \Vert t$. 
\end{siidefinition}
\end{frame}

\subsection{Preliminary Results}
\begin{frame}[triangle=siiblue]
\frametitle{Palindromes}
\begin{siiproposition}[label=palindrome]{}{}
Let $S$ be a set of dice with $|S| = n$ and let $s$ be the string representation of $S$.  If $s = t \Vert t'$ for some string $t$,  then $s$ is $2/n$ permutation fair. 
\end{siiproposition}
\end{frame}

\begin{frame}[triangle=siiblue]
\frametitle{Proof of Proposition \ref{palindrome}}
Because $s = t \Vert t'$ for some $t$, for all $x,y \in s$ we have 
\begin{equation}\nonumber
\begin{split}
(xy)_s &= (xy)_{t \Vert t'} \\
&= (xy)_t + (xy)_{t'} + (x)_s(y)_{t'} \\
&= (xy)_t + \big(m_t^2 - (xy)_t\big) + m_t^2 \\
&= 2m_t^2.
\end{split}
\end{equation}

\vfill

Furthermore, $s = t \Vert t'$ implies that $d_s = 2m_t$. Therefore,
\begin{equation*}
(xy)_s = 2m_t^2 = \frac{m_s^2}{2}.
\end{equation*}
The result then follows from Lemma \ref{perm_fair_counts}. \qed
\end{frame}


\begin{frame}[triangle=siiblue]
\frametitle{Concatenation Theorem}
\begin{siilemma}[label=perm_fair_cat_counts]{}{}
Let $S$ and $T$ be sets of dice with string representations $s$ and $t$ respectively. If $\mathbf{x} \subset s$ with $|\mathbf{x}| = \ell$ then
\begin{equation*}
(\mathbf{x})_{s \Vert t} = \sum_{i=0}^{\ell} (\mathbf{x_{j \leq i}})_s (\mathbf{x}_{j > i})_t
\end{equation*}
\end{siilemma}

\vfill
 
\begin{siitheorem}[label=mn_perm_fair]{Concatenation Theorem}{}
Let $S$ and $T$ be sets of dice. If $S$ and $T$ are $\ell/n$ permutation fair then $S \Vert T$ is $\ell/n$ permutation fair.
\end{siitheorem}
\end{frame}

\begin{frame}[triangle=siiblue]
\frametitle{Proof of Theorem \ref{mn_perm_fair}}
Because $S$ and $T$ are $\ell/n$ permutation fair,  Lemma \ref{perm_fair_counts} implies that for all $0 \leq i \leq \ell$ we have
\begin{equation*}
	(\mathbf{x}_{j\leq i})_s = \frac{m_s^i}{i!}  \qquad and \qquad (\mathbf{x}_{j > i})_t = \frac{m_t^{\ell-i}}{(\ell-i)!}.
\end{equation*}

\vfill

Therefore,  Lemma \ref{perm_fair_cat_counts} implies that
\begin{equation*}
(\mathbf{x})_{s \Vert t} = \frac{1}{\ell!} \sum_{i=0}^\ell \binom{\ell}{i} m_s^i m_t^{\ell-i} = \frac{(m_s + m_t)^\ell}{\ell!}. 
\end{equation*}

The result follows from another application of Lemma \ref{perm_fair_counts}.  \qed
\end{frame}

\begin{frame}[triangle=siiblue]
\frametitle{Relabelling Theorem}
\begin{siitheorem}[label=relabelling]{Relabelling Theorem}{}
Let $S$ be a set of dice with string representation $s$ and $\sigma$ be a permutation on the characters of $s$.  $S$ is $\ell/n$ permutation fair if and only if $\sigma(s)$ is $\ell/n$ permutation fair.
\end{siitheorem}

\end{frame}

\begin{frame}[triangle=siiblue]
\frametitle{Lifting Theorem}
\begin{siitheorem}[label=lifting]{Lifting Theorem}{}
For all $1 \leq i \leq k$ let $S_i$ be a set of $\ell/n$ permutation fair dice with string representation $s_i$. If there exists a constant $C$ such that $\sum (\mathbf{x})_{s_i} = C$ for all $\mathbf{x} \subset s$ with $|\mathbf{x}| = \ell+1$, then $S_1 \Vert S_2 \Vert \ldots \Vert S_k$ is $(\ell+1)/n$ permutation fair.
\end{siitheorem}

\end{frame}
\begin{frame}[triangle=siiblue]
\frametitle{Revisiting $n=3$}
Consider the strings $r = \texttt{abccba}$, $s = \texttt{cabbac}$, and $t = \texttt{bcaacb}$.

\vfill

We have
{
\footnotesize
\begin{table}
\begin{tabular}{c ccc} \toprule
$\mathbf{x}$ & $(\mathbf{x})_r$ & $(\mathbf{x})_s$ & $(\mathbf{x})_t$ \\ \midrule
$\texttt{abc}$ & 2 & 2 & 0 \\
$\texttt{acb}$ & 2 & 0 & 2 \\
$\texttt{bac}$ & 0 & 2 & 2 \\
$\texttt{bca}$ & 2 & 0 & 2 \\
$\texttt{cab}$ & 0 & 2 & 2 \\
$\texttt{cba}$ &  2 & 2 & 0 \\ \bottomrule
\end{tabular}
\end{table}
}
\vfill

So, Theorem \ref{lifting} implies that $r \Vert s \Vert t$ is $3/3$ permutation fair. 
\end{frame}

\begin{frame}[triangle=siiblue]
\frametitle{Revisiting $n=4$}
The Go First String can be written as $r \Vert s \Vert t \Vert t' \Vert s' \Vert r'$ where $r = \texttt{abcddcba}$,  $s = \texttt{dbaccabd}$, and  $t = \texttt{cbaddabc}$.

\vfill

Observe that $r$,  $s$ and $t$ are $2/4$ permutation fair. This is a consequence of both Proposition \ref{palindrome} and Theorem \ref{relabelling}.

\vfill

For all $\mathbf{x}$ with $|\mathbf{x}| = 3$ we have $(\mathbf{x})_r + (\mathbf{x})_s + (\mathbf{x})_t = 36$. Therefore, Theorem \ref{lifting} implies that $r \Vert s \Vert t$ is $3/4$ permutation fair.
\end{frame}

\begin{frame}[triangle=siiblue]
\frametitle{Lifting from $3/n$ to $4/n$}
If we let $v$ = $r \Vert s \Vert t$ then $v$ is $3/4$ permutation fair, $v'$ is $3/4$ permutation fair, and $v \Vert v'$ is the Go First String. The Go First String is $4/4$ permutation fair.

\vfill

In fact, this is an example of a more general phenomenon which we can use to construct a $4/n$ permutation fair string out of any $3/n$ permutation fair string.

\vfill

\begin{siitheorem}[label=lift_34]{}{}
If $n \geq 4$, $t$ is a $3/n$ permutation-fair string, and $s = t \Vert t'$, then $s$ is a $4/n$ permutation-fair string.
\end{siitheorem}
\end{frame}

\begin{frame}[triangle=siiblue]
\frametitle{Proof of Theorem \ref{lift_34}}
Theorem \ref{perm_fair_counts} implies that it suffices to show that for all $\mathbf{x} \subset s$ with $|\mathbf{x}| = 4$ we have $(\mathbf{x})_s = m_s^4 / 24 = 8m_t^4 / 12$. Notice that we have
\begin{equation*}
\begin{split}
(\mathbf{x})_s &= \sum_{i=0}^4 (\mathbf{x_{j \leq i}})_t(\mathbf{x_{j > i}})_{t'} \\
&= (\mathbf{x})_t + (\mathbf{x})_{t'} + \sum_{i=1}^3 (\mathbf{x_{j \leq i}})_t(\mathbf{x_{j > i}})_{t'} \\
&= (\mathbf{x})_t + (\mathbf{x})_{t'}  + \sum_{i=1}^3 \frac{m_t^i}{i!}\frac{m_t^{4-i}}{(4-i)!} \\
&= (\mathbf{x})_t + (\mathbf{x})_{t'}  + \frac{7m_t^4}{12}.
\end{split}
\end{equation*}

So, it suffices to show that $(\mathbf{x})_t + (\mathbf{x})_{t'} = m_t^4 / 12$.
\end{frame}

\begin{frame}[triangle=siiblue]
\frametitle{Proof of Theorem \ref{lift_34} (Continued)}
Let $\mathbf{x} = \texttt{abcd}$ and $\mathbf{x}' = \texttt{dcba}$. Observe that $(\mathbf{x})_{t'} = (\mathbf{x}')_{t}$.

\vfill

Notice that we have
\only<1>{
\begin{equation*}
\begin{split}
m_t^4 \cdot \mathbb{P}\{\texttt{a} < \texttt{b} < \texttt{c}\} =\ &(\texttt{dabc})_t  + (\texttt{adbc})_t + (\texttt{abdc})_t + (\texttt{abcd})_t \\
m_t^4 \cdot \mathbb{P}\{\texttt{a} < \texttt{b}, \texttt{d} < \texttt{c}\} =\  &(\texttt{dabc})_t + (\texttt{adbc})_t + (\texttt{abdc})_t\ + \\
&(\texttt{adcb})_t + (\texttt{dacb})_t + (\texttt{dcab})_t  \\
m_t^4 \cdot \mathbb{P}\{\texttt{d} < \texttt{c} < \texttt{b}\} =\ &(\texttt{adcb})_t + (\texttt{dacb})_t + (\texttt{dcab})_t + (\texttt{dcba})_t
\end{split}
\end{equation*}
}

\only<2->{
\begin{equation*}
\begin{split}
m_t^4 \cdot \mathbb{P}\{\texttt{a} < \texttt{b} < \texttt{c}\} =\ &\renewcommand{\CancelColor}{\color{siiblue}}\cancel{(\texttt{dabc})_t}  + \renewcommand{\CancelColor}{\color{siiorange}}\cancel{(\texttt{adbc})_t} + \renewcommand{\CancelColor}{\color{siiyellow}}\cancel{(\texttt{abdc})_t} + (\texttt{abcd})_t \\
m_t^4 \cdot \mathbb{P}\{\texttt{a} < \texttt{b}, \texttt{d} < \texttt{c}\} =\  &\renewcommand{\CancelColor}{\color{siiblue}}\cancel{(\texttt{dabc})_t} + \renewcommand{\CancelColor}{\color{siiorange}}\cancel{(\texttt{adbc})_t} + \renewcommand{\CancelColor}{\color{siiyellow}}\cancel{(\texttt{abdc})_t}\ + \\
&\renewcommand{\CancelColor}{\color{siiblue}}\bcancel{(\texttt{adcb})_t} + \renewcommand{\CancelColor}{\color{siiorange}}\bcancel{(\texttt{dacb})_t} + \renewcommand{\CancelColor}{\color{siiyellow}}\bcancel{(\texttt{dcab})_t}  \\
m_t^4 \cdot \mathbb{P}\{\texttt{d} < \texttt{c} < \texttt{b}\} =\ &\renewcommand{\CancelColor}{\color{siiblue}}\bcancel{(\texttt{adcb})_t} + \renewcommand{\CancelColor}{\color{siiorange}}\bcancel{(\texttt{dacb})_t} + \renewcommand{\CancelColor}{\color{siiyellow}}\bcancel{(\texttt{dcab})_t} + (\texttt{dcba})_t
\end{split}
\end{equation*}
}

\vfill

Finally, because $\mathbb{P}\{\texttt{a} < \texttt{b} < \texttt{c}\} = \mathbb{P}\{\texttt{d} < \texttt{c} < \texttt{b}\} = 1/6$ and $\mathbb{P}\{\texttt{a} < \texttt{b}, \texttt{d} < \texttt{c}\} = \mathbb{P}\{\texttt{a} < \texttt{b}\}\mathbb{P}\{\texttt{d} < \texttt{c}\} = 1/4$ we conclude that
\begin{align*}
	(\mathbf{x})_t + (\mathbf{x})_{t'}  = (\texttt{abcd})_t + (\texttt{dcba})_t = 2\frac{m_t^4}{6} - \frac{m_t^4}{4}= \frac{m_t^4}{12}
\end{align*}
as required. \qed
\end{frame}

\subsection{New Contstructions}
%\subsubsection{$n=5$}
\begin{frame}[triangle=siiblue]
\frametitle{Tackling $n=5$}
We start with the string
\begin{equation*}
s_5 = \texttt{abcdeedcba}.
\end{equation*}
Notice that $s$ is $2/5$ permutation fair.

\vfill

%As per Theorem \ref{relabelling},  if $\sigma$ is a permutation on the characters $(\texttt{a}, \texttt{b},\texttt{c},\texttt{d},\texttt{e})$ then $\sigma(s)$ is $2/5$ permutation fair as well.
%
%\vfill

If we can find a family of permutations $\{\sigma_i\}_{i=1}^k$ such that $\sum_{i=1}^k (\mathbf{x})_{\sigma_i(s_5)}$ is constant for all $\mathbf{x}$ with $|\mathbf{x}| = 3$, then Theorem \ref{lifting} implies that if
\begin{equation*}
t_5 = \sigma_1(s_5) \Vert \sigma_2(s_5) \Vert \ldots \Vert \sigma_k(s_5)
\end{equation*}
is $3/5$ permutation fair. 
\end{frame}

\begin{frame}[triangle=siiblue]
\frametitle{A Family of Permutations}
It turns out that we can find such a family:
\vfill
\begin{align*}
	\sigma_1 &= \begin{pmatrix}
		\texttt{a} & \texttt{b} & \texttt{c} & \texttt{d} & \texttt{e} \\
		\texttt{a} & \texttt{b} & \texttt{c} & \texttt{d} & \texttt{e}
		\end{pmatrix}
	\qquad
	\sigma_4 = \begin{pmatrix}
		\texttt{a} & \texttt{b} & \texttt{c} & \texttt{d} & \texttt{e} \\
		\texttt{a} & \texttt{c} & \texttt{b} & \texttt{d} & \texttt{e}
		\end{pmatrix} \\ \\
	\sigma_2 &= \begin{pmatrix}
		\texttt{a} & \texttt{b} & \texttt{c} & \texttt{d} & \texttt{e} \\
		\texttt{d} & \texttt{c} & \texttt{b} & \texttt{a} & \texttt{e}
		\end{pmatrix}
	\qquad
	\sigma_5 = \begin{pmatrix}
		\texttt{a} & \texttt{b} & \texttt{c} & \texttt{d} & \texttt{e} \\
		\texttt{d} & \texttt{b} & \texttt{c} & \texttt{a} & \texttt{e}
		\end{pmatrix} \\ \\
	\sigma_3 &= \begin{pmatrix}
		\texttt{a} & \texttt{b} & \texttt{c} & \texttt{d} & \texttt{e} \\
		\texttt{e} & \texttt{b} & \texttt{c} & \texttt{a} & \texttt{d}
		\end{pmatrix}
	\qquad
	\sigma_6 = \begin{pmatrix}
		\texttt{a} & \texttt{b} & \texttt{c} & \texttt{d} & \texttt{e} \\
		\texttt{e} & \texttt{c} & \texttt{b} & \texttt{a} & \texttt{d}
		\end{pmatrix} \\
\end{align*}
\end{frame}

\begin{frame}[triangle=siiblue]
\frametitle{Turning the Crank}
If we let $t_5$ = $\sigma_1(s_5) \Vert \sigma_2(s_5) \Vert \ldots \Vert \sigma_6(s_5)$ then $t_5$ is the string representation of a set of five 12-sided dice which are $3/5$ permutation fair. 

\vfill

Guided by Conjecture \ref{lift_34}, we guess that $u = t_5 \Vert t_5'$ represents a set of five 24-sided dice that are $4/5$ permutation fair.

\vfill

It does!  We find that for all $\mathbf{x}$ with $|\mathbf{x}| = 4$ we have $(\mathbf{x})_u = 24^4 / 24 = 24^3 = 1384$ as per Lemma \ref{perm_fair_counts}.
\end{frame}

\begin{frame}[triangle=siiblue]
\frametitle{An Ugly Finish}
We used a trick to lift our $3/5$ permutation fair dice to a set of $4/5$ permutation fair dice.

\vfill

We haven't found a similar trick that we can apply to efficiently lift that solution to a set of $5/5$ permutation fair dice.

\vfill

The best we've been able to do so far is to let
\begin{equation*}
v = \bigparallel_{\sigma \in S_5} \sigma(u).
\end{equation*}

\vfill

This results in a string that represents a set of five 2880-sided dice that are 5/5 permutation fair.
\end{frame}

%\subsubsection{$n=6$}
\begin{frame}[triangle=siiblue]
\frametitle{Tackling $n=6$}
We start with the string
\begin{equation*}
s_6 = \texttt{abcdeffedcba}.
\end{equation*}
Notice that $s_6$ is $2/6$ permutation fair.

\vfill

If we can find a family of permutations $\{\sigma_i\}_{i=1}^k$ such that $\sum_{i=1}^k (\mathbf{x})_{\sigma_i(s_6)}$ is constant for all $\mathbf{x}$ with $|\mathbf{x}| = 3$, then Theorem \ref{lifting} implies that if
\begin{equation*}
t_6 = \sigma_1(s_6) \Vert \sigma_2(s_6) \Vert \ldots \Vert \sigma_k(s_6)
\end{equation*}
then $t_6$ is $3/6$ permutation fair. 
\end{frame}

\begin{frame}[triangle=siiblue]
\frametitle{Another Family of Permutations}
It turns out that we can find such a family:
\vfill
\begin{align*}
	\sigma_1 &= \begin{pmatrix}
		\texttt{a} & \texttt{b} & \texttt{c} & \texttt{d} & \texttt{e} & \texttt{f} \\
		\texttt{a} & \texttt{b} & \texttt{c} & \texttt{d} & \texttt{e} & \texttt{f}
		\end{pmatrix}
	\qquad
	\sigma_4 = \begin{pmatrix}
		\texttt{a} & \texttt{b} & \texttt{c} & \texttt{d} & \texttt{e} & \texttt{f} \\
		\texttt{a} & \texttt{d} & \texttt{c} & \texttt{b} & \texttt{f} & \texttt{e}
		\end{pmatrix} \\ \\
	\sigma_2 &= \begin{pmatrix}
		\texttt{a} & \texttt{b} & \texttt{c} & \texttt{d} & \texttt{e} & \texttt{f} \\
		\texttt{e} & \texttt{c} & \texttt{b} & \texttt{d} & \texttt{f} & \texttt{a}
		\end{pmatrix}
	\qquad
	\sigma_5 = \begin{pmatrix}
		\texttt{a} & \texttt{b} & \texttt{c} & \texttt{d} & \texttt{e} & \texttt{f} \\
		\texttt{e} & \texttt{d} & \texttt{b} & \texttt{c} & \texttt{a} & \texttt{f}
		\end{pmatrix} \\ \\
	\sigma_3 &= \begin{pmatrix}
		\texttt{a} & \texttt{b} & \texttt{c} & \texttt{d} & \texttt{e} & \texttt{f} \\
		\texttt{f} & \texttt{b} & \texttt{d} & \texttt{c} & \texttt{e} & \texttt{a}
		\end{pmatrix}
	\qquad
	\sigma_6 = \begin{pmatrix}
		\texttt{a} & \texttt{b} & \texttt{c} & \texttt{d} & \texttt{e} & \texttt{f} \\
		\texttt{f} & \texttt{c} & \texttt{d} & \texttt{b} & \texttt{a} & \texttt{e}
		\end{pmatrix} \\
\end{align*}

\end{frame}
\section{Future Work}
\begin{frame}[Triangle=siiorange]
	\tocpage
\end{frame}


\section{References}
\begin{frame}[Triangle=siiorange]
	\tocpage
\end{frame}

\end{document}